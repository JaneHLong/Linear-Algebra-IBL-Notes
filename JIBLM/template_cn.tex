
%%%%%%%%%%%%%%%%%%%%%%%%%%%%%%%%%%%%%%%%%%%%%%%%%%%%%%%%%%%%%%%%%%%
%%%%%%%%%%%%%JIBLM Formatting Package%%%%%%%%%%%%%%%%%%%%%%%%%%%%%%
%%%%%%%%%%%%%Version 1.2: August, 2008%%%%%%%%%%%%%%%%%%%%%%%%%%%
%%%%%%%%%%%%%Author: Paul J. Kapitza, Berry College%%%%%%%%%%%%%%%%
%%%%%%%%%%%%%%%%%%%%%%%%%%%%%%%%%%%%%%%%%%%%%%%%%%%%%%%%%%%%%%%%%%%

\documentclass[oneside]{book}
%%%%%%%%%%%%%journal additions%%%%%%%%%%%%%%%%%%%%%%%%%%%%%%%%%%%%%
\usepackage{time}%make system time available
\usepackage{enumerate}%extended enumeration package
%%%%%%%%%%%%%Symbol libraries%%%%%%%%%%%%%%%%%%%%%%%%%%%%%%%%%%%%%
\usepackage{amssymb}
\usepackage{amsmath}
\usepackage{latexsym}
\usepackage{amsthm}%extended ams-theorem environment

\usepackage{lettrine}%Drop-caps for Masthead
\usepackage{mathptmx}%Times Roman type package for both math and text


\usepackage{endnotes}%Footnotes to the instructor.
   
   
   
   
%%%%%%%%%%%%%Header Customization%%%%%%%%%%%%%%%%%%%%%%%%%%%%%%%%%
\usepackage{fancyhdr}%Header customization
\pagestyle{fancy}
%%%%%%%%%%%%%Chapter headings%%%%%%%%%%%%%%%%%%%%%%%%%%%%%%%%%
\renewcommand{\chaptermark}[1] {\markboth{#1}{}}%

%%%%%%%%%%%%%Page Formatting%%%%%%%%%%%%%%%%%%%%%%%%%%%%%%%%%%
\setlength{\oddsidemargin}{63pt}%%%%%One-sided printing values for 10pt. text-Remove for two sided print
\setlength{\evensidemargin}{63pt}%%%%%One-sided printing values for 10pt. text-Remove for two sided print

\setlength{\parskip}{1mm}
\setlength{\textwidth}{5.0in}
\setlength{\textheight}{8.0in}

%%%%%%%%%%%%%%%%%%%%%%%%%%%%AUTHOR MASTHEAD%%%%%%%%%%%%%%%%%%%%%%%%%%%%%
\newcommand{\authormasthead}{
\begin{flushleft}
\hspace{8.0mm}
\rule{0.3\linewidth}{0.3mm}
\lettrine[lines=2]{D}{raft \rule[3pt]{10mm}{0.5pt} Draft \rule[3pt]{10mm}{0.5pt} Draft \rule[3pt]{10mm}{0.5pt} Draft}
\rule{0.3\linewidth}{0.3mm}
%\hspace{1mm} Issue~\textbf{#1}, Volume #2        Issue 1 (August, 2007)
\vspace{0.2in}
\end{flushleft}
}
%%%%%%%%%%%%%%%%%%%%%%%%%%%%AUTHOR MASTHEAD%%%%%%%%%%%%%%%%%%%%%%%%%%%%%

%%%%%%%%%%%%%%%%%%%%%%%%%%%%TIMESTAMP%%%%%%%%%%%%%%%%%%%%%%%%%%%%%
%%Uses the ``time" package to stamp the time-Editing Feature
\newcommand{\timestamp}{{Edited: \texttt{\now , \today}}}
%%%%%%%%%%%%%%%%%%%%%%%%%%%%TIMESTAMP%%%%%%%%%%%%%%%%%%%%%%%%%%%%%


\let\affiliation\date


%%%%%%%%%%%%%%%%%%%%%%%%%%%% TITLEPAGE%%%%%%%%%%%%%%%%%%%%%%%%%%%%%
%
\makeatletter
\def\maketitle{%
  \null
  \thispagestyle{empty}%
  \timestamp
  \authormasthead
  %\vfill
  \normalfont
  \vspace{2in}
\begin{center}\leavevmode
{\Huge \@title\par}%
\vspace{20mm}
{\Large \@author\par}%
\vspace{5mm}
{\Large \@date\par}% pass affiliation
{\Large \ }
\end{center}
  \vfill
  \null
  \cleardoublepage
 \let\newauthor\@author%transfer to footer line
 }%
\makeatother
%%%%%%%%%%%%%%%%%%%%%%%%%%%% END OF TITLEPAGE%%%%%%%%%%%%%%%%%%%%%%%%%%%%%

%Customized headers and footers- replace authorname with register
\lhead{ \leftmark} \chead{} \rhead{\thepage}
\lfoot{\newauthor} \cfoot{} \rfoot{\emph{DRAFT -- DRAFT -- DRAFT}}
\renewcommand{\headrulewidth}{0.4pt}
\renewcommand{\footrulewidth}{0.4pt}
%
%%%%%%%%%%%%%%%%%%%%%%%%%%%% Annotation Environment %%%%%%%%%%%%%%%%%%%%%%%%%%%%%
\usepackage{comment}
\newcommand{\InstructorVersion}{\includecomment{annotation}}
\newcommand{\StudentVersion}{\excludecomment{annotation}}
%%%%%%%%%%%%%%%%%%%%%%%%%%%% END OF Annotation Environment%%%%%%%%%%%%%%%%%%%%%%%%%%%%%



%%%%%%%%%%%%%%%%%%%%%%%%%%%% Begin--Sectioning Redefines%%%%%%%%%%%%%%%%%%%%%%%%%%%%%
%
\makeatletter
\renewcommand{\@makechapterhead}[1]{%
\vspace*{50\p@}%
  {\parindent \z@ \raggedright \normalfont
    \ifnum \c@secnumdepth >\m@ne
      \if@mainmatter
        \huge \@chapapp\space \thechapter                
        \par\nobreak
        \vskip 20\p@
      \fi
    \fi
    \interlinepenalty\@M
    \LARGE\bfseries  #1\par\nobreak                        
    \vskip 40\p@
  }}


\renewcommand{\@makeschapterhead}[1]{%
  \vspace*{50\p@}%
  {\parindent \z@ \raggedright
    \normalfont
    \interlinepenalty\@M
    \LARGE\bfseries  #1\par\nobreak                      
    \vskip 40\p@
  }}

\makeatother
%%%%%%%%%%%%%%%%%%%%%%%%%%%% End--Sectioning Redefines%%%%%%%%%%%%%%%%%%%%%%%%%%%%%




%%%%%%%%%%Theorem Environments%%%%%%%%%%%%%%%%%%%%%%%%
\newtheorem{theorem}{Theorem}
\newtheorem{acknowledgment}[theorem]{Acknowledgment}
\newtheorem{algorithm}[theorem]{Algorithm}
\newtheorem{axiom}[theorem]{Axiom}
\newtheorem{case}[theorem]{Case}
\newtheorem{claim}[theorem]{Claim}
\newtheorem{conclusion}[theorem]{Conclusion}
\newtheorem{condition}[theorem]{Condition}
\newtheorem{conjecture}[theorem]{Conjecture}
\newtheorem{corollary}[theorem]{Corollary}
\newtheorem{criterion}[theorem]{Criterion}
\newtheorem{definition}[theorem]{Definition}
\newtheorem{example}[theorem]{Example}
\newtheorem{exercise}[theorem]{Exercise}
\newtheorem{lemma}[theorem]{Lemma}
\newtheorem{notation}[theorem]{Notation}
\newtheorem{problem}[theorem]{Problem}
\newtheorem{proposition}[theorem]{Proposition}
\newtheorem{remark}[theorem]{Remark}
\newtheorem{solution}[theorem]{Solution}
\newtheorem{summary}[theorem]{Summary}
%%%%%%%%%%Theorem Environments%%%%%%%%%%%%%%%%%%%%%%%%

%
%%%%%%%%%%%%%%%%%%%%% Annotation Environment Switch%%%%%%%%%%%%
%\StudentVersion
\InstructorVersion
%%%%%%%%%%%%%%%%%%%%% Annotation Environment Switch%%%%%%%%%%%%
%


\begin{document}
\large
\frontmatter
\title{Author's Package }
\author{Paul J. Kapitza \\ Technical Editor}
\affiliation{Georgia Highlands College}
\maketitle
\tableofcontents

\chapter{Preface}

%%%%%%%%%%%%%%%%%%%%%%%%%%%BEGIN REMOVAL {1} %%%%%%%%%%%%%%%%%%%%%%%%%%%%%%%%%
This document provides an introduction to the \textbf{JIBLM} journal package.
As stated under the \emph{Information for Authors} at \emph{www.jiblm.org},
this downloadable template is designed to be self-explanatory,
using standard LaTeX commands and allowing the author to see the formatting exactly as
an accepted manuscript will appear in the journal.  The form \texttt{template.tex} serves two purposes:
\begin{enumerate}
\item It is a guide to typesetting your course notes in JIBLM format using \LaTeX.
\item It is a template to be modified to produce your print ready submission to JIBLM.
\end{enumerate}

\noindent
To get started:
\begin{enumerate}
\item download the Submission Form,
\item download the LaTeX Template,
\item compile \texttt{template.tex} and read the output, and
\item place your notes into a copy of the file \texttt{template.tex}.
\end{enumerate}

\noindent
If you encounter problems, please contact the Technical Editor for assistance;
we welcome feedback.
%%%%%%%%%%%%%%%%%%%%%%%%%%%END REMOVAL {1} %%%%%%%%%%%%%%%%%%%%%%%%%%%%%%%%%




%%%%%%%%%%%%%%%%%%%%%%%%%%%%%%%%%%%%To the Instructor%%%%%%%%%%%%%%%%%%%%%%%%%%%%%%%%

\begin{annotation}
\chapter{To the Instructor}

%%%%%%%%%%%%%%%%%%%%%%%%%%%BEGIN REMOVAL {2} %%%%%%%%%%%%%%%%%%%%%%%%%%%%%%%%%
This is a mandatory introduction to your course notes.  For your convenience, we repeat here
the \emph{Information for Authors} from the website.

Because notes are intentionally brief, it is essential that they be written in a manner that will be usable by other
instructors. For this reason, we ask that you include an introduction that will provide instructors
with a clear understanding of how you have used these notes. This should be a description of the
course you taught, which includes:
\begin{itemize}
\item  its level and role within the broader curriculum,
\item  your expectations and grading of the students,
\item  any unusual characteristics of the class or classroom,
\item  the approximate class size(s), and
\item other information that will be useful to a less-experienced instructor.
A well-written introduction will minimize the instructor's need for further guidance from you.
\end{itemize}

If you choose to use the \textbackslash endnotes feature described in Chapter \ref{ch:annotation} then
you may wish to put a note in this section such as, ``I have added comments on particular problems
throughout the notes.  These superscripts refer to endnotes in the last Chapter titled,
``Notes to the Instructor."

%%%%%%%%%%%%%%%%%%%%%%%%%%%END REMOVAL {2} %%%%%%%%%%%%%%%%%%%%%%%%%%%%%%%%%
\end{annotation}


%%%%%%%%%%%%%%%%%%%%%%%%%%%%%%%%%%%%To the Student%%%%%%%%%%%%%%%%%%%%%%%%%%%%%%%%

\chapter{ To the Student}

%%%%%%%%%%%%%%%%%%%%%%%%%%%BEGIN REMOVAL {3} %%%%%%%%%%%%%%%%%%%%%%%%%%%%%%%%%

This is an optional introduction for the students.  You may add other
optional chapter-like components, such as
\verb|\chapter{Acknowledgments}|.  If you begin a chapter-like
component with \verb|\chapter{}|, the component will be created but
no title will appear.

%%%%%%%%%%%%%%%%%%%%%%%%%%%END REMOVAL {3} %%%%%%%%%%%%%%%%%%%%%%%%%%%%%%%%%



\mainmatter



%%%%%%%%%%%%%%%%%%%%%%%%%%%BEGIN REMOVAL {4} %%%%%%%%%%%%%%%%%%%%%%%%%%%%%%%%%

\chapter{Mandatory Reading}

\section{Getting Started}

The base requirement is  a \LaTeX{} distribution. Once you have this, place the following
three files from \emph{www.jiblm.org/jiblm/info/authorinfo.aspx} into your working directory.

\begin{enumerate}
    \item \texttt{JIBLM.tex} -- This is the file containing additional \LaTeX{} commands
    beyond those of the standard book class in order to typeset a JIBLM document.  You should not need
    to edit this file.
    \item \texttt{template.tex} -- This file contains the text you are currently reading.
    \item A copy of \texttt{template.tex}, named \texttt{myfile.tex} or whatever you like.
\end{enumerate}

\noindent
You are now ready to edit \texttt{myfile.tex} which will become your submission.

\begin{enumerate}

\item At the top of \texttt{myfile.tex}, locate
\begin{verbatim}
  %%% Begin {Title Page} %%%%%%%%%%%%%%%%%%%%%%%%%%%%%%%%%
  \title{Author's Package}
  \author{Paul J. Kapitza}
  \affiliation{Berry College}
  \maketitle
  %%% End {Title Page} %%%%%%%%%%%%%%%%%%%%%%%%%%%%%%%%%%%
\end{verbatim}
and replace ``Author's Package", ``Paul J. Kapitza" and ``Berry
College" with the title of your course notes, your name and your
affiliation.

\item Delete any of the subsequent content of \texttt{myfile.tex} that you wish with
    the exception of the commands \verb|\frontmatter|,
   \verb|\mainmatter|, \verb|\backmatter| and
   \verb|\end{document}| commands. The blocks of statements
   to be removed are easily identified as follows:
\begin{center}
\verb|%%%%%%%%%%%%%%%%%%%%%%BEGIN REMOVAL {n} %%%%%%%%%%%%%%%%%%%%%%|
....material to be removed....
\verb|%%%%%%%%%%%%%%%%%%%%%END REMOVAL {n} %%%%%%%%%%%%%%%%%%%%%%%%%|
\end{center}

\end{enumerate}

Your \texttt{myfile.tex} may now be filled in with your own course notes.

\section{Student and Instructor Versions}\label{ch:annotation}

You may wish to communicate with two different audiences, the students who
will study the materials and the instructors who will teach from the materials.
The \verb|\annotation| environment enables you to create one document that serves
as both a student and an instructor version.  By toggling a single comment in the preamble,
text and mathematics which is encased within the environment will be removed or restored
upon compilation, providing two versions from the same document.  Usage guidelines follow.

\begin{enumerate}


\item To change versions, toggle the comment symbol between the following two lines in \texttt{myfile.tex}.

\begin{center}
\verb|%\StudentVersion|\\
\verb|\InstructorVersion|\\
\end{center}

\item Use the \verb|\annotation| environment by placing opening and closing commands on lines
separate from the material to be annotated, with no starting spaces or characters of any type
on the lines containing the commands.  For example:

\begin{center}
\verb|\begin{annotation}|\\
\verb|... your instructor-specific text goes here ...|\\
\verb|\end{annotation}| \\
\end{center}


\item If a large comment is required, place the text in a separate file and use the commands:

\begin{center}
 \verb|\begin{annotation}|\\
 \verb|\input{filename.tex}|\\
 \verb|\end{annotation}|\\
\end{center}

\item Numbered sequences should be handled with caution
since the automatic numbering of Chapters, Theorems, Equations, etc., is recalculated when
material is removed by the environment.

\item While you may bracket anything you like using the \verb|\annotation{}| environment,
such as comments within the theorem sequence, you may wish to add annotations
to the instructor as footnotes throughout the text which can be made to appear as a chapter-like
component at the end of the document.
\begin{annotation}
\endnote{This is an example of a footnote to the instructor which will appear at the very
end of the document.}
\end{annotation}
This document has such a
component entitled \emph{Notes to the Instructor} which will appear in the
Instructor Version but not in the Student Version.
\begin{annotation}
\endnote{This is another example of a footnote to the instructor, in case you missed the first one.}
\end{annotation}
To do this:
\begin{enumerate}
   \item  Insert in the text itself footnotes to the instructor
       in the following format.
       \begin{verbatim}
   \begin{annotation}
   \endnote{This is an example of a footnote to the instructor.}
   \end{annotation}
       \end{verbatim}
   \item After the command \verb|\backmatter|, add the chapter-like component
       \emph{Notes to the Instructor} that you see following the
       \verb|\backmatter| command in \texttt{template.tex}.  Only do this if you have at
       least one endnote; otherwise you will get an error.
\end{enumerate}
\end{enumerate}



\chapter{Optional Reading}

\section{Theorem-like Environments}

 Theorem environments for declarations are provided and are numbered globally.
As an example of the \verb|\theorem{}| environment consider the following typesetting example.
\begin{theorem}
(The Currant minimax principle.) Let $T$ be completely continuous self adjoint operator
in a Hilbert space $H$. Let $n$ be an arbitrary integer and let $u_1,\ldots,u_{n-1}$ be
an arbitrary system of $n-1$ linearly independent elements of $H$. Denote
\begin{equation}
\max_{\substack{v\in H, v\neq
0\\(v,u_1)=0,\ldots,(v,u_n)=0}}\frac{(Tv,v)}{(v,v)}=m(u_1,\ldots, u_{n-1})
\label{thm:minmax1}
\end{equation}
Then the $n$-th eigenvalue of $T$ is equal to the minimum of these maxima, when
minimizing over all linearly independent systems $u_1,\ldots u_{n-1}$ in $H$,
\begin{equation}
\mu_n = \min_{\substack{u_1,\ldots, u_{n-1}\in H}} m(u_1,\ldots, u_{n-1}) \label{thm:minmax2}
\end{equation}
\end{theorem}
Note: The above equations are automatically numbered as equation (\ref{thm:minmax1}) and
(\ref{thm:minmax2}).



A number of theorem-like environments are included for structuring mathematical statements.  Examples of these are given below in alphabetical order.

\begin{axiom}
This is an axiom
\end{axiom}


\begin{definition}
This is a definition
\end{definition}


\begin{lemma}
This is a lemma
\end{lemma}

\begin{problem}
This is a problem
\end{problem}



\begin{theorem}[Main Theorem]
This is a theorem
\end{theorem}

Additionally, {\bf acknowledgment}, {\bf algorithm}, {\bf case}, {\bf claim}, {\bf conclusion}, {\bf condition},
{\bf conjecture}, {\bf corollary}, {\bf criterion}, {\bf example}, {\bf exercise}, {\bf notation},
{\bf proposition}, {\bf remark}, {\bf solution}, {\bf summary}, and {\bf proof} are available.

\section{Packages, Commands, Styles, and Libraries}

\begin{enumerate}

\item The following packages are used by the Journal:
\begin{itemize}
    \item \texttt{book}--The base book class of \LaTeX.
    \item \texttt{time}--Make system time available.
    \item \texttt{enumerate}--Extended enumeration package.
    \item \texttt{amssymb}, \texttt{amsmath}, \texttt{latexsym}, \texttt{amsthm}--Symbol libraries.
    \item \texttt{lettrine}--Drop-caps.
    \item \texttt{mathptmx}--Times Roman type package for both math and text.
    \item \texttt{fancyhdr}--Header customization.
    \item \texttt{comment}--Base package for the annotation environment.
    \item \texttt{endnotes}--End notes package.
\end{itemize}


\item The current Journal Style does not contain any packages which support graphics content.
To include this facility, place a \verb|\usepackage{pkg-choice}| command  directly after
the first command line in your text file, \verb|%%%%%%%%%%%%%%%%%%%%%%%%%%%%%%%%%%%%%%%%%%%%%%%%%%%%%%%%%%%%%%%%%%%
%%%%%%%%%%%%%JIBLM Formatting Package%%%%%%%%%%%%%%%%%%%%%%%%%%%%%%
%%%%%%%%%%%%%Version 1.2: August, 2008%%%%%%%%%%%%%%%%%%%%%%%%%%%
%%%%%%%%%%%%%Author: Paul J. Kapitza, Berry College%%%%%%%%%%%%%%%%
%%%%%%%%%%%%%%%%%%%%%%%%%%%%%%%%%%%%%%%%%%%%%%%%%%%%%%%%%%%%%%%%%%%

\documentclass[oneside]{book}
%%%%%%%%%%%%%journal additions%%%%%%%%%%%%%%%%%%%%%%%%%%%%%%%%%%%%%
\usepackage{time}%make system time available
\usepackage{enumerate}%extended enumeration package
%%%%%%%%%%%%%Symbol libraries%%%%%%%%%%%%%%%%%%%%%%%%%%%%%%%%%%%%%
\usepackage{amssymb}
\usepackage{amsmath}
\usepackage{latexsym}
\usepackage{amsthm}%extended ams-theorem environment
\usepackage{color}
\usepackage{lettrine}%Drop-caps for Masthead
\usepackage{mathptmx}%Times Roman type package for both math and text
\usepackage[all]{xy}
\usepackage{graphicx}
\usepackage{endnotes}%Footnotes to the instructor.




%%%%%%%%%%%%%Header Customization%%%%%%%%%%%%%%%%%%%%%%%%%%%%%%%%%
\usepackage{fancyhdr}%Header customization
\pagestyle{fancy}
%%%%%%%%%%%%%Chapter headings%%%%%%%%%%%%%%%%%%%%%%%%%%%%%%%%%
\renewcommand{\chaptermark}[1] {\markboth{#1}{}}%

%%%%%%%%%%%%%Page Formatting%%%%%%%%%%%%%%%%%%%%%%%%%%%%%%%%%%
\setlength{\oddsidemargin}{63pt}%%%%%One-sided printing values for 10pt. text-Remove for two sided print
\setlength{\evensidemargin}{63pt}%%%%%One-sided printing values for 10pt. text-Remove for two sided print

\setlength{\parskip}{1mm}
\setlength{\textwidth}{5.0in}
\setlength{\textheight}{8.0in}

%%%%%%%%%%%%%%%%%%%%%%%%%%%%AUTHOR MASTHEAD%%%%%%%%%%%%%%%%%%%%%%%%%%%%%
%\newcommand{\authormasthead}{
%begin{flushleft}
%\hspace{4.4mm}
%\rule{0.3\linewidth}{0.3mm}
%\lettrine[lines=2]{J}{ournal of Inquiry-Based Learning in Mathematics}
%\rule{0.3\linewidth}{0.3mm}
%\hspace{1mm} Issue~\textbf{#1}, Volume #2        Issue 1 (August, 2007)
%\vspace{0.2in}
%\end{flushleft}
%}
%%%%%%%%%%%%%%%%%%%%%%%%%%%%AUTHOR MASTHEAD%%%%%%%%%%%%%%%%%%%%%%%%%%%%%

%%%%%%%%%%%%%%%%%%%%%%%%%%%%TIMESTAMP%%%%%%%%%%%%%%%%%%%%%%%%%%%%%
%%Uses the ``time" package to stamp the time-Editing Feature
\newcommand{\timestamp}{{Edited: \texttt{\now , \today}}}
%%%%%%%%%%%%%%%%%%%%%%%%%%%%TIMESTAMP%%%%%%%%%%%%%%%%%%%%%%%%%%%%%


\let\affiliation\date


%%%%%%%%%%%%%%%%%%%%%%%%%%%% TITLEPAGE%%%%%%%%%%%%%%%%%%%%%%%%%%%%%
%
\makeatletter
\def\maketitle{%
  \null
  \thispagestyle{empty}%
  %\timestamp
  %\authormasthead
  %\vfill
  \normalfont
  \vspace{2in}
\begin{center}\leavevmode
{\Huge \@title\par}%
\vspace{20mm}
{\Large \@author\par}%
\vspace{5mm}
{\Large \@date\par}% pass affiliation
{\Large \ }
\end{center}
  \vfill
  \null
  \cleardoublepage
 \let\newauthor\@author%transfer to footer line
 }%
\makeatother
%%%%%%%%%%%%%%%%%%%%%%%%%%%% END OF TITLEPAGE%%%%%%%%%%%%%%%%%%%%%%%%%%%%%

%Customized headers and footers- replace authorname with register
\lhead{ \leftmark} \chead{} \rhead{\thepage}
\lfoot{\newauthor} \cfoot{} \rfoot{}
\renewcommand{\headrulewidth}{0.4pt}
\renewcommand{\footrulewidth}{0.4pt}
%
%%%%%%%%%%%%%%%%%%%%%%%%%%%% Annotation Environment %%%%%%%%%%%%%%%%%%%%%%%%%%%%%
\usepackage{comment}
\newcommand{\InstructorVersion}{\includecomment{annotation}}
\newcommand{\StudentVersion}{\excludecomment{annotation}}
%%%%%%%%%%%%%%%%%%%%%%%%%%%% END OF Annotation Environment%%%%%%%%%%%%%%%%%%%%%%%%%%%%%



%%%%%%%%%%%%%%%%%%%%%%%%%%%% Begin--Sectioning Redefines%%%%%%%%%%%%%%%%%%%%%%%%%%%%%
%
\makeatletter
\renewcommand{\@makechapterhead}[1]{%
\vspace*{50\p@}%
  {\parindent \z@ \raggedright \normalfont
    \ifnum \c@secnumdepth >\m@ne
      \if@mainmatter
        \huge \@chapapp\space \thechapter
        \par\nobreak
        \vskip 20\p@
      \fi
    \fi
    \interlinepenalty\@M
    \LARGE\bfseries  #1\par\nobreak
    \vskip 40\p@
  }}


\renewcommand{\@makeschapterhead}[1]{%
  \vspace*{50\p@}%
  {\parindent \z@ \raggedright
    \normalfont
    \interlinepenalty\@M
    \LARGE\bfseries  #1\par\nobreak
    \vskip 40\p@
  }}

\makeatother
%%%%%%%%%%%%%%%%%%%%%%%%%%%% End--Sectioning Redefines%%%%%%%%%%%%%%%%%%%%%%%%%%%%%




%%%%%%%%%%Theorem Environments%%%%%%%%%%%%%%%%%%%%%%%%
\newtheorem{theorem}{Theorem}
\newtheorem{acknowledgment}[theorem]{Acknowledgment}
\newtheorem{algorithm}[theorem]{Algorithm}
\newtheorem{axiom}[theorem]{Axiom}
\newtheorem{case}[theorem]{Case}
\newtheorem{claim}[theorem]{Claim}
\newtheorem{conclusion}[theorem]{Conclusion}
\newtheorem{condition}[theorem]{Condition}
\newtheorem{conjecture}[theorem]{Conjecture}
\newtheorem{corollary}[theorem]{Corollary}
\newtheorem{criterion}[theorem]{Criterion}
\newtheorem{definition}[theorem]{Definition}
\newtheorem{example}[theorem]{Example}
\newtheorem{exercise}[theorem]{Exercise}
\newtheorem{question}[theorem]{Question}
\newtheorem{lemma}[theorem]{Lemma}
\newtheorem{notation}[theorem]{Notation}
\newtheorem{problem}[theorem]{Problem}
\newtheorem{proposition}[theorem]{Proposition}
\newtheorem{remark}[theorem]{Remark}
\newtheorem{solution}[theorem]{Solution}
\newtheorem{summary}[theorem]{Summary}
%%%%%%%%%%Theorem Environments%%%%%%%%%%%%%%%%%%%%%%%%
|.  Popular packages available
include \texttt{graphics}, \texttt{graphicx} and \texttt{epsfig}.


\item The following sectioning commands are available:
\begin{itemize}
\item \verb"\chapter{Chapter Title}" command produces a new chapter starting on a new page,
\item \verb"\section{Section Title}" command for major sections, and
\item \verb"\subsection{Subsection Title}" command for subsections.
\end{itemize}

\item Generally, all the styles available from \LaTeX{} are available, including:
\begin{itemize}
    \item Typeset text shapes include \emph{Emphasize}, \textbf{Bold},
\textit{Italics} and \textsl{Slanted} texts.
\item You can also typeset \textrm{Roman}, \textsf{Sans Serif}, \textsc{Small Caps}, and
\texttt{Typewriter} families.
\item The size tags are available;  {\tiny Tiny},
{\scriptsize Scriptsize}, {\footnotesize Footnotesize}, {\small Small}, {\normalsize
Normalsize}, {\large Large 1}, {\Large Large 2 }, {\LARGE Large 3}, \ {\huge Huge 1} and {\Huge
Huge 2}.

\end{itemize}

\item The symbol libraries included within the journal package are the
\texttt{amssymb}, \texttt{amsmath}, \texttt{latexsym} and \texttt{amsthm} packages.
This collection provides for most of the commonly used mathematical typesetting tools to be
directly accessed. Consult your favorite \TeX{} reference for details.   Of particular interest
are the mathematical styles,
\begin{enumerate}
\item $\mathbb{BLACKBOARD}$, (e.g., $\mathbb{R}\,$,$\mathbb{Z}\,$,$\mathbb{C}$).
\item $\mathcal{CALLIGRAPHIC}$, (e.g. $\mathcal{S}= \emptyset$), and
\item $\mathfrak{Fraktur}$.
\end{enumerate}



\end{enumerate}
%%%%%%%%%%%%%%%%%%%%%%%%%%%END REMOVAL {4} %%%%%%%%%%%%%%%%%%%%%%%%%%%%%%%%%


%%%%%%%%%%%%%%%%%%%%%%%%%%%BEGIN REMOVAL {5} %%%%%%%%%%%%%%%%%%%%%%%%%%%%%%%%%
\backmatter

\begin{annotation}
\chapter{Notes to the Instructor}

\renewcommand\notesname{}
\vspace{-2cm}
\begingroup
%\setlength{\parindent}{0pt}% Don't know what this does.  DMC
\setlength{\parskip}{2ex}
\renewcommand{\enotesize}{\normalsize}
\theendnotes
\endgroup
\end{annotation}

\vspace{.1in}

You can also add any text you want here.

%%%%%%%%%%%%%%%%%%%%%%%%%%%END REMOVAL {5} %%%%%%%%%%%%%%%%%%%%%%%%%%%%%%%%%

\end{document}
