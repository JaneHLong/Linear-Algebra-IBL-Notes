\documentclass{exam}
\usepackage[all]{xy}
\usepackage{color,graphicx}
\usepackage{esvect}
\usepackage{amsfonts, amssymb,amsmath}
\usepackage{enumitem}

\makeatletter
\renewcommand*\env@matrix[1][*\c@MaxMatrixCols c]{%
  \hskip -\arraycolsep
  \let\@ifnextchar\new@ifnextchar
  \array{#1}}
\makeatother

\newcommand\Rn{$\mathbb{R}^n$}
\newcommand\Rm{$\mathbb{R}^m$}
\newcommand\R{$\mathbb{R}$}
\newcommand\bq{\begin{question}}
\newcommand\eq{\end{question}}
\newcommand\be{\begin{enumerate}}
\newcommand\ee{\end{enumerate}}
\renewcommand{\labelenumi}{\alph{enumi})}
\newcommand*\colvec[1]{
        \global\colveccount#1
        \begin{bmatrix}
        \colvecnext
}
\def\colvecnext#1{
        #1
        \global\advance\colveccount-1
        \ifnum\colveccount>0
                \\
                \expandafter\colvecnext
        \else
                \end{bmatrix}
        \fi
}

\begin{document}


\textbf{Question 204}: Recall that the set $\{\Vec{0}\}$ is the trivial vector space. What is a basis for $\{\Vec{0}\}$? What is $dim(\{\Vec{0}\})$?\newline
\vspace{0.1in}
\newline

\textit{Basis}: By definition 195, a basis must be a linearly independent. We claim that $c\Vec{0}=\Vec{0}$ for all $c\in\mathbb{R}$: \newline
\vspace{0.1in}
\newline

\textbf{Claim}: For any vector space $V$ and any scalar $c\in\mathbb{R}$, $c\Vec{0}=\Vec{0}$.\newline
\vspace{0.1in}
\newline

\textit{Proof of claim}: Let $c\in\mathbb{R}$ and $\Vec{v}$ be a vector in the vector space $V$. Notice that $V$ must contain at least one vector, since the definition of vector space requires the existence of a zero vector. We have previously shown that $0\Vec{v}=\Vec{0}$. By vector space properties,  $$\Vec{0}=0\Vec{v}=(1+(-1))\Vec{v}=1\Vec{v}+(-1)\Vec{v}=\Vec{v}+(-1)\Vec{v}.$$ Therefore $-\Vec{v}=(-1)\Vec{v}$. That is, the additive inverse of the vector $\Vec{v}$ is equal to the product of the scalar $-1$ and $\Vec{v}$. Therefore 
$$c\Vec{0}=c[\Vec{v}+(-\Vec{v})]=c\Vec{v}+c(-\Vec{v})=c\Vec{v}+c[(-1)\Vec{v}]=c\Vec{v}+[c(-1)]\Vec{v}=[c+c(-1)]\Vec{v}=[1\cdot c+(-1)c]\Vec{v}=[[1+(-1)]c]\Vec{v}=[0c]\Vec{v}=0\cdot\Vec{v}=\Vec{0},$$ as desired.
\vspace{0.1in}
\newline

Returning to the question of a basis for the trivial vector space, it follows from the claim above that $\{\Vec{0}\}$ is a linearly dependent set. Therefore, the only linearly independent subset of $\{\Vec{0}\}$ is the empty set. 

The argument for $span(\emptyset)=\{\Vec{0}\}$ is more difficult to justify, but it is defined in this way. So $\emptyset$ is a basis for $\{\Vec{0}\}$.

\textit{Dimension}: Since the number of elements in $\emptyset$ is zero, the dimension of the vector space $\{\Vec{0}\}$ is 0.

\end{document}